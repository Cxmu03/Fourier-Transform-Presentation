\begin{frame}
	\frametitle{Diskrete Fourier Transformation}
	\begin{itemize}
		\item Fourier Transformation ist ein mathematisches Verfahren um Frequenzen von kontinuierlichen Signalen zu erhalten
		\item Computer arbeiten nur mit diskreten Werten
		\item Wie können zeitdiskrete Signale in ihre Frequenzen zerlegt werden?
	\end{itemize}
\end{frame}

\begin{frame}
	\frametitle{Diskrete Fourier Transformation}
	\begin{align*}
		c_k &= \sum_{n=0}^{N-1}s_n e^{-i2\pi k n / N}
	\end{align*}
	\begin{itemize}
		\item N = Anzahl an Samples
		\item $s_n$ = Sample n
		\item k = Frequenz aus $[0,N-1]$
		\item $c_k$ = DFT mit Information zu Amplitude und Phase
	\end{itemize}
\end{frame}

\begin{frame}
	\frametitle{Inverse DFT}
	\begin{align*}
		s_n &= \frac{1}{N} \sum_{k=0}^{N-1}c_k e^{i2\pi k n/N}
	\end{align*}
\end{frame}